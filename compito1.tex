\documentclass{scrartcl}
\usepackage{physics}
\usepackage{amsmath}
\usepackage{amsfonts}

\renewcommand{\phi}{\varphi}

\begin{document}

\section*{Esercizio 1}
Pongo un sistema di coordinate cartesiane con le \(x\) positive dirette verso la destra del foglio.
Pongo l'origine di questo sistema di coordinate nella posizione della massa all'equilibrio \(x(t=0) = 0\).
Una volta aumentata la costante elastica \(k_2' = 3k\), il secondo principio della dinamica impone la condizione
\begin{equation}
    -k(\ell+x)+ 3k(\ell-x) = m \ddot{x}
\end{equation}
Posso isolare i termini in \(x\) da quelli costanti e definire una nuova quantità \(\omega_0^2 = \frac{k}{m}\)
\[
    m\ddot{x} + 4kx -2k\ell = 0
\]
\begin{equation}
    \ddot{x} + 4 \omega_0^2 x = 2 \omega_0^2\ell
\end{equation}
Risolvo l'equazione omogenea 
\[
    \lambda^2 +4\omega_0^2 = 0 \longrightarrow \lambda = \pm 2 \omega_0 i
\]
\begin{equation}
    x(t) = A\cos(2w_0t) + B\sin(2w_0t) + \overline{x}(t)
\end{equation}
dove \(\overline{x}(t)\) rappresenta la soluzione particolare che andremo a cercare tra le funzioni del tipo \(f(t) = \gamma\) costante.
Siccome \(\dv{\gamma}{t} = 0 \quad \dv[2]{\gamma}{t} = 0\), sostituendo all'interno della (3)
\[4\omega_0^2 \gamma = 2\omega_0^2 \ell \longrightarrow \gamma = \frac{1}{2}\ell\]
Da cui otteniamo
\begin{equation}
    x(t) = A\cos(2w_0t) + B\sin(2w_0t) + \frac{1}{2}\ell
\end{equation}
Impongo le condizioni iniziali \(x(t=0) = 0\) e \(\dot{x}(t=0) = 0\) per trovare i coefficienti \(A\) e \(B\)
\[x(t=0) = A +\frac{1}{2}\ell = 0 \longrightarrow A = \frac{1}{2}\ell\]
\[\dot x(t)\eval_{0} = \frac{1}{2}\ell \sin(2\omega_0 t) + B \cos(2\omega_0 t) \eval_{0} = 0 = B\]
La soluzione della (2) è dunque 
\begin{equation}
    x(t) = \frac{1}{2}\ell \cos(2\omega_0 t) = \frac{1}{2}\ell \qty[1-\cos(2\omega_0 t)]
\end{equation}
Se la calcolo in \(t^* = \frac{\pi}{4}\sqrt{\frac{m}{k}}\) ottengo
\begin{equation}
    x(t^*) = \frac{1}{2}\ell \qty[1-\cos(\frac{\pi}{2})] = \frac{1}{2}\ell
\end{equation}

\section*{Esercizio 2}
\subsection*{Parte 1}
\setcounter{equation}{0}
    Pongo un sistema di coordinate cartesiane con l'asse \(y\) rivolto verso il basso.
    Imposto il sistema di equazioni differenziali
    \begin{equation}
        \begin{cases}
            m \ddot \phi_1 = -k \phi -k(\phi_1 - \phi_2) + mg \\
            m \ddot \phi_2 = -k (\phi_2 - \phi_1)
        \end{cases}
    \end{equation}
    Definisco \(\omega_0^2 := \frac{k}{m}\) e riscrivo il sistema portando in forma normale.
    \begin{equation}
        \begin{cases}
            \ddot \phi_1 - \omega_0^2 \qty(2 \phi_1 - \phi_2) = g \\
            \ddot \phi_2 - \omega_0^2 \qty(\phi_2 - \phi_1) = g
        \end{cases}
    \end{equation}
    Per trovare i moti normali cerco le soluzioni al sistema del tipo \(\phi_1 = A e^{i\omega t}\) e \(\phi_2 = B e^{i \omega t}\) con \(A, B \in \mathbb{C}\).
    Calcolo le derivate seconde \(\ddot \phi_1 = - \omega^2 \phi_1\) e \(\ddot \phi_2 = - \omega^2 \phi_2\) e sostituisco nel sistema (2) rimuovendo la costante \(g\) per risolvere il sistema omogeneo.
    \begin{equation}
        \begin{cases}
            - \omega^2 A e^{i \omega t} = - \omega_0^2 \qty(2A -B) e^{i \omega t} \\
            - \omega^2 B e^{i \omega t} = - \omega_0^2 \qty(B-A) e^{i \omega t}
        \end{cases}
    \end{equation}
    Posso dividere tutto per \(e^{i \omega t}\) siccome non è mai nullo e definire \(\lambda = \frac{\omega^2}{\omega_0^2}\) per ottenere il sistema
    \begin{equation}
        \begin{cases}
            \lambda A = 2A-B \\
            \lambda B = B-A
        \end{cases}
    \end{equation}
    Che può essere riscritto come
    \begin{equation}
        \lambda \begin{bmatrix} A \\ B \end{bmatrix} = \begin{bmatrix} 2 & -1 \\ -1 & 1 \end{bmatrix} \begin{bmatrix} A \\ B \end{bmatrix} = M \begin{bmatrix} A \\ B \end{bmatrix}
    \end{equation}
    Ciò equivale a trovare gli autovalori della matrice \(M\). I rispettivi autovettori saranno due soluzioni linearmente indipendenti del sistema, il che ci permette di trovarle tutte a meno di combinazione lineare.
    \begin{equation}
        \lambda_1 = \frac{3-\sqrt{5}}{2} \quad u_1 = \begin{bmatrix} \frac{\sqrt{5}-1}{2} \\ 1 \end{bmatrix} \qquad \lambda_2 = \frac{3+\sqrt{5}}{2} \quad u_2 = \begin{bmatrix} - \frac{\sqrt{5}+1}{2} \\  1\end{bmatrix}
    \end{equation}
    Ciò significa che il sistema fisico ha due moti normali linearmente indipendenti, rispettivamente con pulsazioni
    \[\omega_1 = \omega_0 \sqrt{\frac{3-\sqrt{5}}{2}} \qquad \omega_2 = \omega_0 \sqrt{\frac{3+\sqrt{5}}{2}}\]
    Cerco ora una soluzione particolare che, per il criterio di somiglianza sarà del tipo \(\Gamma = \begin{bmatrix}\gamma_1 \\ \gamma_2\end{bmatrix} = \) costante, dunque con derivate nulle. Sostituisco tale \(\Gamma\) nel sistema (2) per ottenere
    \begin{equation}
        \begin{cases}
            2 \gamma_1 -\gamma_2 = \frac{g}{\omega_0^2} \\
            - \gamma_1 + \gamma_2 = \frac{g}{\omega_0^2}
        \end{cases}
    \end{equation}
    Il che mi porta alla soluzione \(\Gamma = \begin{bmatrix}2 \\ 3\end{bmatrix} \frac{g}{\omega_0^2}\).
    Il moto generico del sistema fisico avrà dunque equazione
    \begin{equation}
        \Phi = \alpha \Phi_1 + \beta \Phi_2 + \Gamma 
    \end{equation}
    \[\Phi = \mqty[\phi_1 \\ \phi_2] \qquad \Phi_1 = u_1 e^{i \omega_1 t} \qquad \Phi_2 = u_2 e^{i \omega_2 t}\]

    \subsection*{Parte 2}
    Per trovare le condizioni iniziali prima cerco le \(\phi_1^{eq}\) e \(\phi_2^{eq}\) ossia gli scostamenti delle due masse dallo zero all'equilibrio del sistema, ponendo nulla l'accellerazione nel sistema (2).
    \begin{equation}
        \begin{cases}
            mg - k\phi_1 + k(\phi_2-\phi_1) = 0 \\
            mg - k(\phi_2-\phi_1) = 0
        \end{cases}
    \end{equation}
    Questo è un sistema lineare che, definendo \(\omega_0^2 = \frac{k}{m}\) porta alle soluzioni:
    \[\phi_1^{eq} = 2\frac{g}{\omega_0^2} \qquad \phi_2^{eq} = 3\frac{g}{\omega_0^2}\]
    Confrontando con la soluzione generale del moto del sistema, si può notare come le posizioni d'equilibrio e la soluzione particolare siano uguali. 
    Ciò è un buon segno in quanto la matematica ci dice che questa formulazione equivale ad un'oscillazione delle due masse attorno alla loro posizione di equilibrio, il che ha fisicamente senso.
    Ponendo come condizioni iniziali \(\phi_{1, 0} = \phi_1^{eq}\) e \(\phi_{2, 0} = \phi_2^{eq} + 1 \text{cm}\) all'interno della soluzione generale (8) ricaviamo i valori di \(\alpha\) e \(\beta\) per questo particolare set-up del sistema.
    \[\alpha = \frac{\sqrt{5} - 5}{10} \qquad \beta = \frac{5 - \sqrt{5}}{10}\]
    Sostituendo questi valori nell'equazione (8) otteniamo quindi
    \begin{equation}
        \mqty[\phi_1 \\ \phi_2] = \frac{\sqrt{5}-5}{10} \mqty[\frac{\sqrt{5}-1}{2} \\ 1] e^{it\omega_0\sqrt{\frac{3-\sqrt{5}}{2}}} + \frac{5-\sqrt{5}}{10} \mqty[-\frac{1+\sqrt{5}}{2} \\ 1] e^{it\omega_0\sqrt{\frac{3+\sqrt{5}}{2}}}
    \end{equation}
    Infine imponiamo che la velocità iniziale del sistema sia nulla prendendo solo la parte reale di questa equazione, ottenendo la legge oraria del sistema 
    \begin{equation}
        \mqty[\phi_1 \\ \phi_2] = \frac{\sqrt{5}-5}{10} \mqty[\frac{\sqrt{5}-1}{2} \\ 1] \cos({t\omega_0\sqrt{\frac{3-\sqrt{5}}{2}}}) + \frac{5-\sqrt{5}}{10} \mqty[-\frac{1+\sqrt{5}}{2} \\ 1] \cos({t\omega_0\sqrt{\frac{3+\sqrt{5}}{2}}})
    \end{equation}
\end{document}